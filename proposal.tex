\documentclass{article}
\usepackage{fullpage}
\title{From Interaction Trees to Asynchronous Tests}
\author{Yishuai Li}
\begin{document}
\maketitle
\section{Introduction}
Importance of rigorous testing in software development.

Main challenge: nondeterminism in various parts of the system.

Goal: a systematic testing framework for networked systems, addressing internal
nondeterminism and network nondeterminism.

Methodology: interpret model specification into testing logic; introduce
abstract representation for test input generation and shrinking.

Broader impact: Handling network nondeterminism is the first step towards
testing generic concurrent systems.  Automatically deriving tests from formal
specifications enables iterative incremental development of reliable software.

\section{Background}
Property-based testing with QuickChick.  Limitation: nondeterminism makes
validation logic hard to write.

Model-based testing in ISSTA paper and its related work.
\section{Research Plan}
\subsection{Test input generation and shrinking}
Interesting test inputs might depend on runtime observations of the system.  To
represent such dependency, I'll propose an abstract language for test inputs.
The language enables generating new inputs based on previous observations during
test.  In particular, it can handle partial observation caused by internal and
network nondeterminism.

\subsection{Generic specification language}
Based on my experiments on various web applications, I'll propose a generic
library for specifying application-level protocols.  Developers use the
domain-specific language to define protocol-specific aspects {\it e.g.}
prototype logic, message encoding {\it etc.}  and the library automatically
derives the specification into an interactive tester client that can reveal
servers' incomformance effectively.

\end{document}
