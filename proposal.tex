\documentclass{article}
\usepackage{fullpage}
\usepackage[dvipsnames,svgnames,x11names]{xcolor}
\usepackage{listings,lstparams,lstcoq}
\lstdefinestyle{customcoq}{
    language=Coq
}
\newcommand{\ilc}[1]{\lstinline[style=customcoq]{#1}}

\title{From Interaction Trees to Asynchronous Tests}
\author{Yishuai Li}
\begin{document}
\maketitle
\section{Introduction}
Importance of rigorous testing in software development.

Main challenge: nondeterminism in various parts of the system.

Goal: a systematic testing framework for networked systems, addressing internal
nondeterminism and network nondeterminism.

Methodology: interpret model specification into testing logic; introduce
abstract representation for test input generation and shrinking.

Broader impact: Handling network nondeterminism is the first step towards
testing generic concurrent systems.  Automatically deriving tests from formal
specifications enables iterative incremental development of reliable software.

Todo thesis claim.  What technology solves what problem.

5--6 pages.

\section{Background}
Property-based testing with QuickChick.  Limitation: nondeterminism makes
validation logic hard to write.

ISSTA related work.

5--6+ pages.

\section{Prior Results}
Model-based testing in ISSTA paper.

Refactor previous paper.

\section{Research Plan}
\subsection{Test Input Generation and Shrinking}

\subsubsection{Tester language}
A tester is a client program that performs \ilc{send} and \ilc{recv} interactions.

Interesting test inputs might depend on runtime observations of the system.  To
represent such dependency, I'll propose an abstract language for test inputs.
The language enables generating new inputs based on previous observations during
test.  In particular, it can handle partial observation caused by internal and
network nondeterminism.

\subsection{Generic specification language}
Based on my experiments on various web applications, I'll propose a generic
library for specifying application-level protocols.  Developers use the
domain-specific language to define protocol-specific aspects {\it e.g.}
prototype logic, message encoding {\it etc.}  and the library automatically
derives the specification into an interactive tester client that can reveal
servers' incomformance effectively.

5--6+ pages.  Include technical details for evaluation.

Timeline towards dissertation.  Month-level schedule, including publication.

\end{document}
